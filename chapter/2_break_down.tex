\chapter{Break Things Down}

\begin{quotation}
  经过反复的体验, 我认识到: 57年前悬崖上的那一次使人记忆深刻的教训, 使我在之后的生活中能够正确对待诸如看得太远, 想得太多, 瞻前顾后, 灰心丧气等不利心理因素. 我时常告诫自己: 不要看那些离得很远的岩石, 只管迈出第一步;站稳脚跟之后, 再迈另一步, 直至达到预期的目标. 这时, 再回过头来看我所走过的路程, 我就会为自己所取得的成就感到惊讶和自豪. 

  \emph{莫顿·亨特《悬崖上的一课》}\footnote{这段话摘自小学课本《走一步再走一步》的原文. 不过都过了这么久了, 我也真的不记得这段课文到底是在哪里出现过了. 笑. }
\end{quotation}

好的, 同学们, 让我们把课本翻到, 呃, 随便哪一页吧. 读一页是一页. 就像是如何击败葫芦娃一样, 千万别一口气打七个, 把那座大山锯开逐个击破就好了. 

\section{切一半, 再切一半}
\begin{quotation}
  当我通过防盗门的时候, 警铃大作. 那个壮硕的图书管理员把我拦了下来, 将我的书分成了两半, 一半放到防盗门上, 没响, 然后是另外一半. 

  \emph{不记得究竟是在哪里读到这段故事了}\footnote{我记得《孤儿列车》里面有类似的情节. \\ 或者我记得应该是一个笑话? 大概是主人公借书的时候刷漏了, 然后在一本一本地去测的时候就被边上的管理员一同操作秀到了的故事. \\ 不管它, 我们会在后面发现这个检测的过程中, 体现了一种二分法的算法. }
\end{quotation}

曾记否, 好几页之前, 