\chapter{Preface}

\centerline{I'm boring. }

我不过就是一个业余的业余计算机爱好者, 不会什么很硬核的东西和技能, 想要尝试写这个教程不过是自己的无聊想法而已. 基本上是在边学边写. 主要的写作动力来自于我的朋友们和我的计算机科学导论课程. 

所以在最前面我应该声明一点: 我的这本教程应该定位为一个普及性质的东西, 并且和严格的科班教材应该还是有区别的. 呃, 怎么说呢, 也不能这么说, 
毕竟我还是参考了许多的书, 并且你可以说, 某些部分我就是基本照抄的, (因为写得太好了, 并且我也只是刚学而已). \footnote{不过放心, 我不会像是某些编程语言的书一样, 从差劲的hello world开始, 然后开始无情的分类介绍, 什么字符串操作, 什么变量赋值, 什么控制流之类的. 虽然这样会比较系统, 但是对于计算机是什么以及如何看待计算机好像没什么用. }

但是毕竟这个教程是边学边写, 并且我是一个很随心所欲的菜狗. 所以感觉我的风格可能就像是那种从前互联网时代的计算机爱好者, 吃饱了撑着在网络上漫游, 然后不知不觉就不小心学到了知识... 学的东西不是很成体系, 并且还很支离破碎. 可能会想到哪就写到哪. 所以我在这里先立下(反向)flag, 大概我想要讲的内容和东西如下: 

\begin{itemize}
  \item \textbf{计算机语言} 因为我只稍微接触过一点点 Ruby, 
  (ぎりぎり勉强能用的水平). 所以我将以它为主要的介绍对象. 
  \footnote{其实选择 Ruby 语言的还有一个原因是在学习它的过程中, 我遇到了一个很厉害的家伙: \_why, 他的 Why's (poignant) Guide to Ruby 对我的影响真的很大. 呃, 算是夹带私货了吧. 并且我这本书将会尽可能地去学习why先生的讲故事的风格. }
  \item \textbf{计算机玩具} 嗯, 为了留住我的读者, 
  我的朋友不是理科的, 然后她/他/它好摆烂(bushi, 我什么都没说), 
  不是很喜欢玩这种不好玩的东西. 
  \item \textbf{程序逆向和计算机构造} 因为我就是通过这些才学会的,
  计算机... 不要不信啊. 
  \item \dots
\end{itemize}

嗯, 毕竟是草稿性质的东西, 写到哪里算哪里. 又, 我的文章不过是拾人牙慧, 山寨他人思想的普通文字罢了. 

(害怕, 我们的院长说他整了好几年还是没能够写完一本原子物理, 可见写一本教材的困难, 难道我现在正面对着世界上的一个极其困难的任务? )

% 成品要删除这一段: 
% 目前的打算: 
% 
% 第一章说明计算机程序设计的思想: 就是将大的过程分解成小的过程, 将小的过程组合成大的过程. 整本书就只有这一个概念, 其他的不过是这个概念的点缀而已. 顺带教一点点Ruby. 
% 
% 第二章计算机是一种约定主义. 
% 
% 第三章算法. % 高德纳? 
% 
% 第四章