\chapter{Preface}

\centerline{I'm boring. }

我不过就是一个业余的业余计算机爱好者, 不会什么很硬核的东西和技能, 想要尝试写这个教程不过是自己的无聊想法而已. 基本上是在边学边写. 主要的写作动力来自于我的朋友们和我的计算机科学导论课程. 

所以在最前面我应该声明一点: 我的这本教程应该定位为一个普及性质的东西, 并且和严格的科班教材应该还是有区别的. 呃, 怎么说呢, 也不能这么说, 
毕竟我还是参考了许多的书, 并且你可以说, 某些部分我就是基本照抄的, (因为写得太好了, 并且我也只是刚学而已). \footnote{不过放心, 我不会像是某些编程语言的书一样, 从差劲的hello world开始, 然后开始无情的分类介绍, 什么字符串操作, 什么变量赋值, 什么控制流之类的. 虽然这样会比较系统, 但是对于计算机是什么以及如何看待计算机好像没什么用. }

但是毕竟这个教程是边学边写, 并且我是一个很随心所欲的菜狗. 所以感觉我的风格可能就像是那种从前互联网时代的计算机爱好者, 吃饱了撑着在网络上漫游, 然后不知不觉就不小心学到了知识... 学的东西不是很成体系, 并且还很支离破碎. 可能会想到哪就写到哪. 所以我在这里先立下(反向)flag, 大概我想要讲的内容和东西如下: 

\begin{itemize}
  \item \textbf{计算机语言} 因为我只稍微接触过一点点 Ruby, 
  (ぎりぎり勉强能用的水平). 所以我将以它为主要的介绍对象. 
  \footnote{其实选择 Ruby 语言的还有一个原因是在学习它的过程中, 我遇到了一个很厉害的家伙: \_why, 他的 Why's (poignant) Guide to Ruby 对我的影响真的很大. 呃, 算是夹带私货了吧. 并且我这本书将会尽可能地去学习why先生的讲故事的风格. }
  \item \textbf{计算机玩具} 嗯, 为了留住我的读者, 
  我的朋友不是理科的, 然后她/他/它好摆烂(bushi, 我什么都没说), 
  不是很喜欢玩这种不好玩的东西. 
  \item \textbf{程序逆向和计算机构造} 因为我就是通过这些才学会的,
  计算机... 不要不信啊. 
  \item \dots
\end{itemize}

嗯, 毕竟是草稿性质的东西, 写到哪里算哪里. 又, 我的文章不过是拾人牙慧, 山寨他人思想的普通文字罢了. 

(害怕, 我们的院长说他整了好几年还是没能够写完一本原子物理, 可见写一本教材的困难, 难道我现在正面对着世界上的一个极其困难的任务? )

% 成品要删除这一段: 
% 目前的打算: 
% 
% 第一章说明计算机程序设计的思想: 就是将大的过程分解成小的过程, 将小的过程组合成大的过程. 整本书就只有这一个概念, 其他的不过是这个概念的点缀而已. 顺带教一点点Ruby. 
% 
% 第二章计算机是一种约定主义. 
% 
% 第三章算法. % 高德纳? 
% 
% 第四章

\subsection*{如何阅读本书}
\begin{quotation}
  要由这本书来传达的只是一个单一的思想, 可是尽管我费劲心力, 除了用这全本的书以外, 还是不能发现什么捷径来传达这一思想. 

  \emph{叔本华《作为意志和表象的世界》 (石冲白 译)}
\end{quotation}

啊哈哈, 因为这本书有些像是在和我的朋友对话的过程中写成的, 也就是说, 其中的内容是非常零碎和比较混乱的(就是一本注释可能比正文还要长的书了, 并且里面除了计算机编程, 还乱七八糟地扯了很多的别的东西). 所以对于阅读本书的读者来说, 可能会有一种"这家伙讲的东西怎么和一般的东西不太一样啊! "的想法. 

我的特点就是废话特别多, 所以还请谅解, 如果你实在是看不下去的话, 请跳过我的那些废话吧(笑). 我想要构造的教程应该是这样的一个"思想的系统"(参见叔本华的《作为意志和表象的世界》第一版序): 在结构上相互关联, 在整体中涵蕴部分而又部分中体现整体. "整个思想通过各个部分而显明, 而不预先理解全部, 也不能彻底了解任何最细微的部分". 

也许你会觉得这样不就会导致"为了学会A, 就要学会B, 但是为了学会B, 却要学会A"这样无厘头的矛盾循环吗? 所以为了防止这样的困境, 我打算在一开始就简单地介绍贯穿全书的概念"抽象". 然后在这之后来不断反复地应用这个概念. 

\subsection*{全书的写作大纲}
这个是我在写作的时候起草的一个大纲性质的东西, 不过希望也能够给你阅读的感觉: 

\subsubsection*{第一章 - 介绍语言}
\begin{itemize}
  \item 抽象是... (参考John Locke)
  \item 形式地介绍一门编程语言: Ruby
  \item 稍微介绍形式上的含义, 用Ruby来理解Ruby, 但是不是那么严格的解释
  \item 写代码的一些规范
\end{itemize}

\subsubsection*{第二章 - 分解过程}
\begin{itemize}
  \item 用Ruby来模拟逻辑运算: 计算的过程是子过程的组合
  \item 用Ruby来模拟一个计算器
  \item 用Ruby模拟一个简单图灵机, 模拟一个计算机
  \item 如何将"分解的过程"用过程来描述
\end{itemize}

\subsubsection*{第三章 - 组合过程}
\begin{itemize}
  \item 如何将小过程组合成大过程
  \item 数据的流动
\end{itemize}

\subsubsection*{第四章 - 描述过程的算法}
\begin{itemize}
  \item 
\end{itemize}